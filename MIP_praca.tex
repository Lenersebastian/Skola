% Metódy inžinierskej práce

\documentclass[10pt,twoside,slovak,a4paper]{article}

\usepackage[slovak]{babel}
\usepackage[IL2]{fontenc}
\usepackage[utf8]{inputenc}
\usepackage{graphicx}
\usepackage{url} % príkaz \url na formátovanie URL
\usepackage{hyperref}
\usepackage{cite}
%\usepackage{times}

\pagestyle{headings}

\title{Ako hranie hier ovplyvňuje učenie vekovej skupiny od 10 do 15 rokov\thanks{Spresnenie rámcovej témy, vedenie: MSc. Mirwais Ahmadzai}}

\author{Sebastian Lener\\[2pt]
{\small Slovenská technická univerzita v Bratislave}\\
{\small Fakulta informatiky a informačných technológií}\\
{\small \texttt{xlener@stuba.sk}}
}

\date{\small 8. október 2022}



\begin{document}
\maketitle
\begin{abstract}
\ldots
\end{abstract}



\section{Intro}
Už od malička vieme, že nadmerné hranie videohier má na nás zlý vplyv, spôsobuje závislosti a v niektorých prípadoch zvyšuje agresivitu. Môžu mať avšak hry vplyv aj na našu inteligenciu a schopnosť sa učiť? V tomto článku sa budem venovať tomu, ako ovplyvňujú schopnosť sústrediť sa, riešiť problémy, zlepšujú (v niektorých prípadoch možno zhoršujú) krátkodobú alebo aj dlhodobú pamäť, zlepšujú v cudzích jazykoch, pomáhajú študentom s vývinovými poruchami (napríklad dyslexia) a ďalším faktorom, ktoré na nás hry majú. Túto tému som si vybral, pretože ma veľmi zaujíma ako hry ovplyvňujú našu psychológiu a každodenný život. 

Zdroje: https://www.youtube.com/watch?v=OOsqkQytHOs
\section{este uvidim}


\bibliography{literatura.bib}
\bibliographystyle{plain}

\end{document}
